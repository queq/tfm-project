\capitulo{1}{Introducción}

En un mundo cada vez más dependiente de las tecnologías de la información y comunicaciones, el acontecimiento de ciberataques trae mayores riesgos y posible impacto negativo en las sociedades. Como evidencia, 2023 fue un año en que ocurrieron importantes filtraciones de datos masivas, e incidentes que comprometieron el funcionamiento de entidades de ámbito nacional \cite{moore2023}.

Entre las amenazas principales, el \textit{phishing} se ha mantenido como el vector de acceso inicial más común por su inherente relación con la comunicación humana, y por la diversidad de métodos de ataque, los cuales se han vuelto más sofisticados ante la aparición de nuevos avances, como la IA generativa \cite{enisa2023}.

El medio más usual para transmitir ataques de \textit{phishing} es a través de páginas web, y como mecanismo de detección se emplean desde métodos simples como listas negras, hasta los más sofisticados, que a partir de técnicas de \textit{Machine Learning} o Inteligencia Artificial, realizan inferencias sobre la legitimidad del contenido, basándose en características de la página como (1) la URL; (2) el dominio; (3) metainformación del código fuente; (4) aspecto visual de la página; o (5) una combinación de las anteriores \cite{castaño2021}.

