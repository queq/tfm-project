\documentclass[a4paper,12pt,oneside,oldfontcommands]{memoir}

% Castellano
% es-ucroman: evitar el warning por Fake sc para números romanos en minúscula
\usepackage[spanish,es-tabla,es-ucroman]{babel}
\selectlanguage{spanish}
\usepackage[utf8]{inputenc}
\usepackage[T1]{fontenc}
\usepackage{lmodern} % Scalable font
\usepackage{microtype}
\usepackage{placeins}
\usepackage{gensymb}

\RequirePackage{booktabs}
\RequirePackage[table]{xcolor}
\RequirePackage{xtab}
\RequirePackage{multirow}

% Links
\usepackage[colorlinks]{hyperref}
\hypersetup{
	allcolors = {blue}
}

% Ecuaciones y simbolos (tanto AMS como LaTeX
\usepackage{amsmath,amssymb,latexsym}

% Rutas de fichero / paquete
\newcommand{\ruta}[1]{{\sffamily #1}}

% Párrafos
\nonzeroparskip

% Huérfanas y viudas
\widowpenalty100000
\clubpenalty100000

% Paper size and margins (para A4)
\settrimmedsize{297mm}{210mm}{*}
\setlength{\trimtop}{0pt}
\setlength{\trimedge}{\stockwidth}
\addtolength{\trimedge}{-\paperwidth}
\settypeblocksize{634pt}{448.13pt}{*}
\setulmargins{4cm}{*}{*}
\setlrmarginsandblock{2.5cm}{2.5cm}{*}
\setmarginnotes{36pt}{36pt}{\onelineskip}
\setheadfoot{\onelineskip}{2\onelineskip}
\setheaderspaces{*}{2\onelineskip}{*}
\checkandfixthelayout

\input{miiutiles}

% Evitar solapes en el header
\nouppercaseheads

% Datos de portada
\title{Herramientas y comparativas para el estudio de métodos de detección de
URLs \textit{phishing}}
\newcommand{\nombre}[0]{SERGIO AGUDELO BERNAL} %%% cambio de comando
\newcommand{\depto}[0]{DEPARTAMENTO DEL TUTOR}
\newcommand{\areac}[0]{AREA\_CONOCIMIENTO DEL TUTOR}
\author{\nombre}
\tutor{JESÚS MARÍA VEGAS HERNÁNDEZ}
\date{\today}

\begin{document}

\maketitle


\newpage\null\thispagestyle{empty}\newpage


%%%%%%%%%%%%%%%%%%%%%%%%%%%%%%%%%%%%%%%%%%%%%%%%%%%%%%%%%%%%%%%%%%%%%%%%%%%%%%%%%%%%%%%%
\thispagestyle{empty}


\noindent
\begin{center}%
	{\noindent\Huge Universidad de Valladolid}\vspace{.5cm}%

	\begin{center}%
		\includegraphics[height=3cm]{img/escudoUVA} \hspace{1cm}
	\end{center}%

	{\noindent\Large \textbf{Máster universitario en Ingeniería Informática}}\vspace{.5cm}%
\end{center}%



\noindent D. \makeatletter\@tutor\makeatother, profesor del departamento de \depto, área de \areac.

\noindent \textbf{Expone}:

\noindent Que el alumno D. \nombre, %con DNI dni,
ha realizado el Trabajo final de Máster en Ingeniería Informática titulado "\makeatletter\textsc{\@title{}}\makeatother".

\noindent Y que dicho trabajo ha sido realizado por el alumno bajo la dirección del que suscribe, en virtud de lo cual se autoriza su presentación y defensa.

\begin{center} %\large
	En Valladolid, {\large \today}
\end{center}

\vfill\vfill\vfill

% Author and supervisor
% \begin{minipage}{0.45\textwidth}
% 	\begin{flushleft} %\large
% 		V\degree. B\degree. del Tutor:\\[2cm]
% 		D. nombre tutor
% 	\end{flushleft}
% \end{minipage}
% \hfill
% \begin{minipage}{0.45\textwidth}
% 	\begin{flushleft} %\large
% 		V\degree. B\degree. del co-tutor:\\[2cm]
% 		D. nombre co-tutor
% 	\end{flushleft}
% \end{minipage}
% \hfill

% \vfill

% para casos con solo un tutor comentar lo anterior
% y descomentar lo siguiente
\begin{minipage}{0.45\textwidth}
	\begin{flushleft} %\large
		V\degree. B\degree. del Tutor:\\[2cm]
		Jesús María Vegas Hernández
	\end{flushleft}
\end{minipage}


\newpage\null\thispagestyle{empty}\newpage




\frontmatter

% Abstract en castellano
\renewcommand*\abstractname{Resumen}
\begin{abstract}
	En este primer apartado se hace una \textbf{breve} presentación del tema que se aborda en el proyecto.
\end{abstract}

\renewcommand*\abstractname{Descriptores}
\begin{abstract}
	Palabras separadas por comas que identifiquen el contenido del proyecto Ej: servidor web, buscador de vuelos, android \ldots
\end{abstract}

\clearpage

% Abstract en inglés
\renewcommand*\abstractname{Abstract}
\begin{abstract}
	A \textbf{brief} presentation of the topic addressed in the project.
\end{abstract}

\renewcommand*\abstractname{Keywords}
\begin{abstract}
	keywords separated by commas.
\end{abstract}

\clearpage

% Indices
\tableofcontents
\clearpage
\listoffigures
\clearpage
\listoftables
\clearpage
\mainmatter

\capitulo{1}{Introducción}

En un mundo cada vez más dependiente de las tecnologías de la información y comunicaciones, el acontecimiento de ciberataques trae mayores riesgos y posible impacto negativo en las sociedades. Como evidencia, 2023 fue un año en que ocurrieron importantes filtraciones de datos masivas, e incidentes que comprometieron el funcionamiento de entidades de ámbito nacional \cite{moore2023}.

Entre las amenazas principales, el \textit{phishing} se ha mantenido como el vector de acceso inicial más común por su inherente relación con la comunicación humana, y por la diversidad de métodos de ataque, los cuales se han vuelto más sofisticados ante la aparición de nuevos avances, como la IA generativa \cite{enisa2023}.

El medio más usual para transmitir ataques de \textit{phishing} es a través de páginas web, y como mecanismo de detección se emplean desde métodos simples como listas negras, hasta los más sofisticados, que a partir de técnicas de \textit{Machine Learning} o Inteligencia Artificial, realizan inferencias sobre la legitimidad del contenido, basándose en características de la página como (1) la URL; (2) el dominio; (3) metainformación del código fuente; (4) aspecto visual de la página; o (5) una combinación de las anteriores \cite{castaño2021}.


\include{./tex/2_Objetivos_del_proyecto}
\include{./tex/3_Conceptos_teoricos}
\include{./tex/4_Tecnicas_y_herramientas}
\include{./tex/5_Aspectos_relevantes_del_desarrollo_del_proyecto}
\include{./tex/6_Trabajos_relacionados}
\include{./tex/7_Conclusiones_Lineas_de_trabajo_futuras}


%\renewcommand\chaptername{Anexo}
%\renewcommand\thechapter{\Roman{chapter}}
%\setcounter{chapter}{0}

% Añadir entrada en el índice: Anexos
\appendix
\addcontentsline{toc}{part}{Apéndices}
\part*{Apéndices}

\include{./tex/A_Plan_proyecto}
\apendice{Especificación de Requisitos}

Si el TFM comporta el desarrollo de software, en este apéndice se reunirá la Especificación de Requisitos del mismo.

\section{Introducción}

\section{Objetivos generales}

\section{Catalogo de requisitos}

\section{Especificación de requisitos}



\apendice{Documento de Diseño}

Si el TFM comporta el desarrollo de software, en este apéndice se reunirá el Documento de Diseño asociado al mismo.

\section{Introducción}

\section{Diseño de datos}

\section{Diseño procedimental}

\section{Diseño arquitectónico}



\include{./tex/D_Manual_programador}
\apendice{Documentación de usuario}

En este apéndice se incluye la documentación necesaria para que el usuario sepa cómo debe instalar y usar el sistema desarrollado.

\section{Introducción}

\section{Requisitos de usuarios}

\section{Instalación}

\section{Manual del usuario}





\bibliographystyle{acm}
\bibliography{bibliografia}

\end{document}
